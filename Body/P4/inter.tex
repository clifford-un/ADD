Una de las características de la aplicación es la capacidad de intercambiar información con otros sistemas y/o componentes de software de manera que sus funcionalidades puedan ser potenciadas, complementadas o delegadas mediante la comunicación entre sistemas, bajo este principio Kwii Platform expone cada una de sus funcionalidades de manera controlada para el uso por parte de otros sistemas, e incluso por parte de otros componentes de la misma aplicación, a la vez que cuenta con la capacidad de usar información proveniente de otras aplicaciones de software.

\subsection{Uso del estilo REST como estándar de comunicación}
La vía principal de comunicación de Kwii Platform es el protocolo de comunicación REST, el cual es usado por su simpleza y expresividad. Su uso está extendido por todos los niveles de la aplicación, en particular la comunicación entre el componente orquestador y todos los micro-servicios que conforman la aplicación. Adicionalmente, haciendo uso del lenguaje de
consulta GraphQL se potencia la capacidad de intercambio de información con otros componentes de la plataforma como sus componentes de FronEnd.

\subsection{Uso de SOAP como alternativa de comunicación}
Como complemento al protocolo REST, se usa SOAP como un protocolo alternativo de comunicación con sistemas externos a través del uso de el lenguaje de marcado XML, lo cual provee total independencia a los detalles de funcionamiento de dichos sistemas manteniendo una integridad semántica en el uso e intercambio de información, pudiendo así ser usados por Kwii Platform.